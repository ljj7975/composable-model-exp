\documentclass{article}
\usepackage[nonatbib]{nips_2016}

\usepackage[breaklinks=true,letterpaper=true,colorlinks,citecolor=black,bookmarks=false]{hyperref}

\usepackage{amsthm}
\usepackage{amsmath,amssymb}
\usepackage{enumitem}

\usepackage[sort&compress,numbers]{natbib}
\usepackage[normalem]{ulem}

% use Times
\usepackage{times}
% For figures
\usepackage{graphicx} % more modern
%\usepackage{epsfig} % less modern
%\usepackage{subfig} 

\graphicspath{{../fig/}}

\usepackage{tikz}
\usepackage{tkz-tab}
\usepackage{caption} 
\usepackage{subcaption} 
\usetikzlibrary{shapes.geometric, arrows}
\tikzstyle{arrow} = [very thick,->,>=stealth]

\usepackage{cleveref}
\usepackage{setspace}
\usepackage{wrapfig}
%\usepackage[ruled]{algorithm}
\usepackage{algpseudocode}
\usepackage[noend,linesnumbered]{algorithm2e}

\usepackage[disable]{todonotes}

\usepackage{algpseudocode}

\title{Dynamic Model Construction for Efficient Classification}

\author{
	Jaejun Lee \\
	School of Computer Science\\
	University of Waterloo\\
	Waterloo, ON, N2L 3G1 \\
	\texttt{j474lee@uwaterloo.ca} \\
}

\begin{document}
\maketitle

\begin{abstract} 

One of the major drawback of neural network in the domain of classification is that retraining is unavoidable when the target set of class changes. For this reason, networks are often designed to be wide and deep. However, this leads to increase the necessary computations which has a direct impact on the efficiency of the model. In this work, I study how different combination of loss function and last activation function affects the classification output and present a new way to add or remove a class from target class set without retrianing. With this technique, a model can be adjusted to classify any combination of target classes and the minimal resource usage is guaranteed as the adjusted model involves the same amount of computations as the model trained to classify the same set of class explicitly.

\end{abstract} 

\section{Introduction}
In this section you are going to present a brief background and motivation of your project. Why is it interesting/significant? How does it relate to the course?

\section{Related Works}

\subsection{Ensemble Learning}

\subsection{Multi-task Learning}

\subsection{Transfer Learning}

\section{Proposed Works}

similar idea has been proposed to share most of the compuations where separate fc layers has been proposed

taking the same idea, train on individual class, load the weight to construct the target



\section{Activation and Loss}
Perform an initial review of relevant literature. Has your problem, or one of similar nature, been considered before? By whom? What are the differences or limitations (if any)?

\subsection{How many class can it predict}

\section{Experiments}

\subsection{implementation}

\subsection{results}
Perform an initial review of relevant literature. Has your problem, or one of similar nature, been considered before? By whom? What are the differences or limitations (if any)?

\subsection{computational efficiency}
Perform an initial review of relevant literature. Has your problem, or one of similar nature, been considered before? By whom? What are the differences or limitations (if any)?

\section{Conclusion}
In this section please concisely describe what you are going to achieve in this project. E.g., formulate your problem precisely (mathematically), present the technical challenges (if any), discuss the tools or datasets that you will build on, state your goals, and come up with a plan for evaluation.

For your own sake, you might want to lay out a time line, so that you can keep a good track of your project.

\newpage

\section*{Acknowledgement}
Thank people who have helped or influenced you in this project.

\nocite{*}

\bibliographystyle{unsrtnat}
\bibliography{project}

\end{document}